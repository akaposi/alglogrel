\documentclass{lmcs} %%% last changed 2014-08-20

%% mandatory lists of keywords 
\keywords{MANDATORY list of keywords}

%% read in additional TeX-packages or personal macros here:
%% e.g. \usepackage{tikz}
\usepackage{hyperref}
%%\input{myMacros.tex}
%% define non-standard environments BEYOND the ones already supplied 
%% here, for example
\theoremstyle{plain}\newtheorem{satz}[thm]{Satz} %\crefname{satz}{Satz}{S\"atze}
%% Do NOT replace the proclamation environments lready provided by
%% your own.

\def\eg{{\em e.g.}}
\def\cf{{\em cf.}}

%% due to the dependence on amsart.cls, \begin{document} has to occur
%% BEFORE the title and author information:

\begin{document}

\title[Short title TODO]{Homomorphisms from logical relations}
\titlecomment{{\lsuper*} This is a journal version of the paper \cite{hiit}.}

\author[A.~Kaposi]{Ambrus Kaposi}	%required
\address{Department of Programming Languages and Compilers, E{\"o}tv{\"o}s Lor{\'a}nd University, Budapest, Hungary}	%required
\email{akaposi@inf.elte.hu}  %optional
%\thanks{thanks 1, optional.}	%optional

\author[A.~Kov{\'a}cs]{Andr{\'a}s Kov{\'a}cs}	%optional
\address{Department of Programming Languages and Compilers, E{\"o}tv{\"o}s Lor{\'a}nd University, Budapest, Hungary}	%required
\email{kovacsandras@inf.elte.hu}  %optional
\thanks{This work was supported by the European Union, co-financed by the
European Social Fund (EFOP-3.6.3-VEKOP-16-2017-00002) and COST Action
EUTypes CA15123.}	%optional

%% etc.

%% required for running head on odd and even pages, use suitable
%% abbreviations in case of long titles and many authors:

%%%%%%%%%%%%%%%%%%%%%%%%%%%%%%%%%%%%%%%%%%%%%%%%%%%%%%%%%%%%%%%%%%%%%%%%%%%

%% the abstract has to PRECEDE the command \maketitle:
%% be sure not to issue the \maketitle command twice!

\begin{abstract}
  \noindent If a logical relation between two algebras is a graph of a
  function, it is an algebra homomorphism. 
\end{abstract}

\maketitle

\section*{Introduction}

In HoTT, algebraic structures can have higher equalities. What is a
homomorphism between two such algebraic structures? Example: circle,
sphere. This also answers the question: what is an elimination
principle for any HIT?

Another Motivation: usefulness of GATs for formalising programming
languages. Example: tt-in-tt with equations. Algebraic view of
progr. langs.

Method: Uday observation (inverse Reynolds), signatures for
generalised algebraic theories using domain-specific tt, parametricity
interpretation of dependent types. We put these
together. Achievements: getting induction principles for non-strictly
positive types, describing induction principles for HIITs.

This is a journal version, main new thing: relation to parametricity
(graph model), category model (extension of reflexive graph model),
homomorphisms for non strictly positive operators, relation of
syntactic translations and models for solving coherence issue, for
HIITs the syntactic translation cannot be extended to a category.

Formalisation?

\subsection*{Overview of the paper}

\subsection*{Related work}

\section*{Theory of signatures}

Brick red color. Subst.calc, U,El,Pi. We formalise the calculus as a
QIIT, but present it with variables names and judgements. Eliminating
from the QIIT needs to preserve equalities. Metatheory extensional.

\section*{From logical relations to homomorphisms}

Graph model, (A,M''), check what it gives for natural numbers

we restrict U from relations to functions and El becomes graph of the
function

now we get homomorphisms

example: homomorphism for non-strictly positive operators

problem: suc is not the usual one, we cannot compose homomorphisms:
composition does not work for Pi

\section*{Categories of algebras}

Restriction to strictly positive operators in signatures.

category model

\section*{Extensions}

Pi with metatheoretic domain and large codomain

Pi with metatheoretic domain and small codomain (infinitary)

Extensional identity

Identities between sorts

Examples

\section*{Higher equalities}

The metatheory has K, so we do not elimiante into it, but to a target
theory which does not have K. This also solves a coherence problem
when eliminating from the syntax as a QIIT in HoTT as
metatheory. Also, we add intensional Id type.

\subsection*{Source theory}

(Brick red)

\subsection*{Target theory}

(Blue)

\subsection*{Translations}

cannot be extended to a category because U[] is not strict unless we
add new computation rules.

\subsection*{Examples}

\section*{Conclusions and further work}

\section*{Acknowledgment}
  \noindent The authors wish to acknowledge fruitful discussions with
  A and B.


\bibliographystyle{plain}
\bibliography{b}
  
\appendix
\section{}

appendix contents

\end{document}
